% !TEX root = ../main.tex
\chapter{致\ZJUspace{}谢}


行文至此,光标停留在这一页的末尾,窗外的求是园似乎也比往日更加静谧。这篇关于操作系统快照优化的论文,不仅是对我硕士生涯的一个交代,更像是一场漫长的、与系统底层机制死磕的独家记忆。回首这两年多的时光,从最初面对 QEMU 庞大代码库时的茫然无措,到如今能从内核态到用户态构建起 HPRO 的整套逻辑,这一路虽无惊天动地的波澜,却充满了在此刻回味起来显得尤为珍贵的点滴。

首先,我要向我的导师赵新奎老师致以最深沉的敬意。在学术上,赵新奎老师有着近乎苛刻的严谨,每当我试图用模糊的“大概”来解释实验数据时,他总能敏锐地指出逻辑的断层。是他教会我,在资源受限的系统里,每一个字节的节省、每一毫秒的优化都值得被认真对待。但他又不只是严师,更像是一位耐心的引路人,在我因“脏页追踪”机制陷入瓶颈而焦虑失眠时,是他办公室深夜的灯光和鼓励,让我重拾了解决问题的信心。师恩如海,铭记于心。

感谢我的同门与实验室的伙伴们,特别是叶淼、欧阳创宇、邓宇真和李浩德。我们共享了实验室无数个日夜的键盘敲击声,也共享了调试代码时遇到 Segmentation Fault 的抓狂与实验曲线终于收敛时的狂喜。那些关于写放大、关于锁竞争的激烈讨论,往往比书本上的理论更能击中问题的本质。谢谢你们,让原本枯燥的科研生活变得生动且有温度。这段并肩作战的情谊,将是我毕业囊中极为厚重的行囊。

我要特别感谢我的父母。求学二十余载,家始终是我最坚实的后盾。你们或许并不了解我的研究内容,但你们给予我的无条件支持——无论是物质上的保障还是精神上的宽容,都是我能心无旁骛地在异乡求学的底气。你们的爱与包容,是我在任何受限环境下都能通过“快照”恢复满状态的源动力。

此外,还要感谢在论文盲审与答辩过程中提出宝贵意见的专家学者们,你们的质疑与建议,让这项研究更加完善,也让我看到了自身视野的局限。

最后,我想谢谢那个在无数个深夜里没有选择放弃的自己。感谢那个在面对 Flash 存储特性时反复碰壁却依然坚持推导的自己。硕士生涯虽短,但这种在受限条件下寻求最优解的思维方式,将伴随我走向未来更广阔的山海。

凡是过往,皆为序章。带着浙大赋予我的求是精神,我将整理行装,奔赴下一场旅程。
