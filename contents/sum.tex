%!TEX root = ../main.tex

\chapter{总结和展望}

\section{总结}

本文的研究源于边缘计算快速发展背景下的一个核心工程矛盾:即日益复杂的边缘智能应用对系统高可用性的需求,与边缘设备极其有限的物理资源(算力、带宽、存储寿命)之间的冲突。在资源充裕的数据中心环境中表现优异的传统虚拟化快照技术,一旦下沉至边缘侧,便因其高昂的资源开销和静态的调度策略而难以为继。

针对这一问题,本文并未沿用传统的参数调优思路,而是从系统架构层面重新思考了快照机制与底层硬件资源的交互模式,设计并实现了 HPRO 系统。回顾全文,本研究的主要工作与结论概括如下:

第一,构建了基于硬件辅助的低开销热点感知机制,解决了“监测精度”与“系统损耗”的权衡难题。针对传统写保护机制在低主频处理器上引发“中断风暴”的问题,本文提出利用处理器页表项中的硬件脏位作为零开销的写入传感器,并结合自适应多位老化算法对内存访问模式进行量化。实验表明,这种非侵入式的软硬协同方案,能够以极低的 CPU 占用率精准识别出内存中的核心工作集与瞬时热点,确立了“仅保存必要数据”的优化前提。

第二,提出了基于资源反馈的动态决策模型,实现了快照策略从“静态执行”向“自适应调度”的转变。本文认为,边缘环境的负载特征是高度动态且异构的。为此,研究定义了系统压力指数(SPI),将 CPU 负载、I/O 队列深度及内存水位等多维指标映射为统一的量化信号。基于此信号,HPRO 能够在高压、标准与低压模式之间进行实时切换,动态调整热点判定的阈值与后台保存的速率。这种机制确保了系统在业务繁忙时能够主动让渡资源以保障 QoS,而在资源闲置时则能高效利用带宽,从而实现了资源利用效率的最大化。

第三,设计了面向 Flash 特性的存储优化方案,缓解了 I/O 瓶颈与介质磨损问题。针对嵌入式存储设备读写不对称及写放大效应显著的物理特性,本文在执行层引入了环形缓冲区写合并、多层级去重及细粒度提取技术。通过将上层离散的逻辑写入请求重组为底层的顺序物理写入,并配合基于 SPI 的令牌桶流控机制,HPRO 有效降低了快照操作对存储总线的瞬时冲击,在保证数据一致性的同时,显著延长了设备的存储寿命。

\section{展望}

尽管 HPRO 系统在当前的实验环境下验证了其有效性,但作为一项探索性研究,本文在设计深度与广度上仍存在局限。随着边缘计算硬件架构的演进与应用场景的拓展,以下几个方向值得在未来工作中进行更深入的探讨:

首先,异构加速器的状态一致性问题亟待解决。本文的研究范畴主要局限于 CPU 与内存状态的快照。然而,当前的边缘智能设备日益依赖 GPU、NPU 或 FPGA 等专用加速器进行推理加速。这些加速器内部拥有独立的寄存器、显存及流水线状态,目前的通用快照机制难以对其进行透明的保存与恢复。如何在不中断业务逻辑的前提下,实现包含异构硬件状态的全系统一致性快照,是未来实现边缘 AI 任务无缝迁移的关键。

其次是新型存储与互连架构的适配性研究。 本研究主要针对主流的 NAND Flash 特性进行优化。然而,随着 CXL(Compute Express Link)技术的落地以及非易失性内存(NVM)的逐步商用,内存与存储之间的界限正在模糊。未来的边缘架构可能具备统一寻址的持久化内存空间,这将从根本上改变现有的“内存-磁盘”数据拷贝范式。探索基于新硬件特性的零拷贝快照甚至原地持久化机制,将具有重要的学术价值。

最后是从单节点优化向分布式协同演进。 HPRO 目前聚焦于单机环境下的资源博弈。但在工业控制或车联网等场景中,业务往往跨多个节点分布式部署。当单一节点执行快照或回滚时,如何通过轻量级的分布式协议保证集群内其他节点的状态协同,避免因状态不一致引发的逻辑错误,是构建高可用边缘集群必须面对的挑战。

未来的研究工作将致力于在上述方向继续拓展,以期为资源受限环境下的系统软件优化提供更完善的理论支撑与技术方案。