% !TEX root = ../main.tex

% 定义中文摘要和关键字
\begin{cabstract}
随着人工智能技术、边缘计算以及轻量化云计算平台的迅速发展,大量模型推理、实时控制和复杂计算任务开始被部署在资源受限的异构环境中。这些环境往往以虚拟化技术为基础,通过轻量级虚拟机或隔离的运行时实例承载业务逻辑,使其在云边协同、无人系统、工业控制和分布式智能等场景中获得广泛应用。在此背景下,操作系统快照作为保障系统一致性、支持在线迁移、容错恢复和任务重构的关键技术,其最主要的实现形式即为虚拟化平台中的虚拟机状态快照。然而传统快照技术是在资源充裕的云服务器条件下发展成熟的,当其下沉到计算性能有限、内存紧张、存储带宽受限甚至能源敏感的资源受限环境时,预拷贝(Pre-Copy)、后拷贝(Post-Copy)等经典机制面临严重的性能退化:脏页产生速率高于传输速率导致迭代难以收敛,高频 VM Exit 与缺页处理引入额外延迟,大量后台写入造成 Flash 写放大与 I/O 阻塞,并在 AI 推理场景下进一步放大系统抖动与实时性违约风险。

针对上述挑战,本研究提出了一种基于热点感知与系统资源反馈的操作系统快照优化框架,并重点面向虚拟化环境中的内存快照流程进行设计。该方法利用硬件辅助脏位构建轻量级自适应多位老化模型,以极低开销捕捉虚拟机内存写入的时空局部性,精准识别热点区域;通过定义系统压力指数(System Pressure Index, SPI),实时感知 CPU、I/O 队列深度、内存水位与能源状态,从而自适应地调节停机阶段的页面选择范围、后台迁移的节奏与增量写入的批处理策略。此外,本研究针对资源受限环境常用的 Flash 存储介质特性,设计了写合并、细粒度去重和基于令牌桶的 I/O 节流机制,以减少写放大、降低突发带宽占用并提升整体传输效率。

在基于 QEMU/KVM 的原型系统中验证结果表明,该方法能够在保持虚拟机状态强一致性的前提下,显著降低快照过程对 AI 推理与实时业务的干扰,缩短停机时间,并减少快照数据量与存储开销。研究成果为构建面向边缘智能与资源受限环境的高可用虚拟化基础设施提供了高效、可扩展的快照优化方案。
\end{cabstract}

\ckeywords{操作系统快照,虚拟化快照,热点感知,资源受限环境}
